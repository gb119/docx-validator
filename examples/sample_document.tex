\documentclass{article}

\usepackage{amsmath}
\usepackage{graphicx}
\usepackage{hyperref}

\title{Sample LaTeX Document}
\author{Jane Smith}
\date{\today}

\begin{document}

\maketitle

\begin{abstract}
This is a sample LaTeX document that demonstrates the structure expected by the validator. It includes sections, figures, equations, and citations.
\end{abstract}

\section{Introduction}

This document serves as a test case for the LaTeX parser in the docx-validator tool. It demonstrates various LaTeX features that can be validated.

\section{Document Structure}

A well-structured LaTeX document should have proper sections, subsections, and use appropriate environments for different content types.

\subsection{Sections and Subsections}

Use \texttt{\textbackslash section} and \texttt{\textbackslash subsection} commands to create a hierarchical structure.

\subsection{Mathematical Content}

LaTeX excels at typesetting mathematical content. Here's an inline equation: $E = mc^2$.

And here's a displayed equation:

\begin{equation}
    \int_{-\infty}^{\infty} e^{-x^2} dx = \sqrt{\pi}
    \label{eq:gaussian}
\end{equation}

We can reference Equation~\ref{eq:gaussian} later in the document.

\section{Figures and Tables}

\subsection{Figures}

Documents often include figures to illustrate concepts:

\begin{figure}[h]
    \centering
    % \includegraphics{example-image}
    \caption{An example figure caption}
    \label{fig:example}
\end{figure}

As shown in Figure~\ref{fig:example}, visual elements enhance understanding.

\subsection{Tables}

Tables can present structured data:

\begin{table}[h]
    \centering
    \begin{tabular}{|l|l|}
        \hline
        \textbf{Feature} & \textbf{Description} \\
        \hline
        Validation & LLM-based document validation \\
        Multiple Formats & Supports DOCX, HTML, and LaTeX \\
        \hline
    \end{tabular}
    \caption{Features of docx-validator}
    \label{tab:features}
\end{table}

Table~\ref{tab:features} summarizes the key features.

\section{Citations and Bibliography}

Academic documents often include citations \cite{knuth1984texbook, lamport1994latex}. These references help readers find additional information.

\subsection{Cross-references}

LaTeX's cross-referencing system allows you to reference equations, figures, tables, and sections automatically. This ensures consistency throughout the document.

\section{Conclusion}

This sample document demonstrates various LaTeX elements that can be validated using the docx-validator tool. The tool can analyze document structure, verify the presence of required elements, and ensure consistency.

\begin{thebibliography}{9}

\bibitem{knuth1984texbook}
Donald E. Knuth,
\textit{The TeXbook},
Addison-Wesley, 1984.

\bibitem{lamport1994latex}
Leslie Lamport,
\textit{LaTeX: A Document Preparation System},
Addison-Wesley, 1994.

\end{thebibliography}

\end{document}
